\documentclass[./../main.tex]{subfiles}
\graphicspath{{img/}}

\begin{document}
    \color{blue}
    \begin{exercise}[Extra: Velocidades en el Modelo de Electrones Libres]
        Considera un metal en donde los electrones de conducción son descritos por el modelo de electrones libres. Recordemos que en este modelo, la velocidad promedio de los electrones en la superficie de Fermi está dada por la velocidad de Fermi \(v_{F} = \slashfrac{\hbar k_{F}}{m}\), en donde \(k_{F}\) es el vector de onda de Fermi.

        \begin{enumerate}
            \item (Valor: +1pt) - Usando los resultados obtenidos en clase, demuestra que la \textbf{velocidad de arrastre} de un electrón en presencia de un campo eléctrico \(\vect{E}\) está dada por
            
            \begin{equation}
                \vect{v}_{a} = -\dfrac{\sigma\vect{E}}{n\e}
                \label{eq:DriftSpeed}
            \end{equation}

            en donde \(\sigma\) es la conductividad eléctrica. Demuestra también que \(\sigma\) en términos del camino libre medio \(\ell\) está dada por

            \begin{equation}
                \sigma = \dfrac{n\e^{2}\ell}{mv_{F}}.
                \label{eq:Conductivity}
            \end{equation}

            \color{black}
            \begin{solution}
                Sabemos que la densidad de corriente está dada en términos de la velocidad de arrastre como

                \begin{equation}
                    \vect{J} = -n\e\vect{v}_{a}.
                    \label{eq:CurrentDensity}
                \end{equation}

                Y, por la ley de Ohm, tenemos que \(\vect{J} \propto \vect{E}\), \idest

                \begin{equation}
                    \vect{J} = \sigma\vect{E},
                    \label{eq:OhmLawCurrentDensity}
                \end{equation}

                Igualando estas expresiones y resolviendo para \(\vect{v}_{a}\),

                \begin{align*}
                    -n\e\vect{v}_{a} &= \sigma\vect{E},\\
                    \Aboxedmain{\vect{v}_{a} &= -\dfrac{\sigma\vect{E}}{n\e}.}
                \end{align*}

                Por otro lado, en clase se obtuvo que la velocidad de arrastre para un electrón con momento \(\vect{p}\) es

                \begin{equation}
                    \vect{v}_{a} =  \dfrac{-\e E\tau}{m},\\
                    \label{eq:DriftSpeedWithRelaxationTime}
                \end{equation}

                con \(\tau\) el tiempo de relajación.

                Sustituyendo \cref{eq:DriftSpeedWithRelaxationTime} en \cref{eq:CurrentDensity}, la densidad de corriente queda como

                \begin{align*}
                    \vect{J} &= -n\e\left(\dfrac{-\e\vect{E}\tau}{m}\right),\\
                    \vect{J} &= \dfrac{n\e^{2}\tau}{m}\vect{E}.
                \end{align*}

                Y comparando su magnitud con la de \cref{eq:OhmLawCurrentDensity}, vemos que

                \begin{empheq}[box=\secbox]{equation*}
                    \sigma = \dfrac{n\e^{2}\tau}{m}.
                \end{empheq}

                Sabemos además que la velocidad típica de un electrón es \(v_{F}\) por lo que el camino libre medio es

                \begin{align*}
                    \ell &= v_{F}\tau,\\
                    \implies\tau &= \dfrac{\ell}{v_{F}}.
                \end{align*}

                Por lo tanto,

                \begin{empheq}[box=\resultbox]{equation*}
                    \sigma = \dfrac{n\e^{2}\ell}{mv_{F}}.
                \end{empheq}
            \end{solution}

            \color{blue}
            \item (Valor: +1pt) - Considera un alambre hecho de cobre:
                \begin{enumerate}[label = (b.\arabic*)]
                    \item Calcula el valor de \(v_{a}\) y de \(v_{F}\) suponiendo que la temperatura del alambre es de \qty{300}{\kelvin} y se le aplica un campo eléctrico cuya magnitud es \(E = \qty{1}{\V\per\m}\). Comenta sobre cómo se comparan ambas velocidades.
                    
                    \color{black}
                    \begin{solution}
                        Recordamos que la velocidad de Fermi \(v_{F}\) está dada como

                        \begin{equation}
                            v_{F} = \dfrac{\hbar k_{F}}{m},
                            \label{eq:FermiVelocity}
                        \end{equation}

                        con \(k_{F} = (3\pi^{2}n)^{\slashfrac{1}{3}}\) el vector de onda de Fermi.

                        Por un lado, de \cref{eq:DriftSpeed} tenemos que la magnitud de la velocidad de arrastre \(v_{a}\) es

                        \begin{align*}
                            v_{a} = \dfrac{\sigma E}{n\e} &= \dfrac{(\qty[per-mode=power]{5.9e7}{\per\ohm\per\m})(\qty{1}{\V\per\m})}{(\qty[per-mode=symbol]{8.45e28}{\atoms\per\m\cubed})(\qty{1.6e-19}{\C})},\\
                            \Aboxedmain{v_{a} &\simeq \qty[per-mode=symbol]{4.36e-3}{\m\per\s}.}
                        \end{align*}

                        Mientras que de \cref{eq:FermiVelocity}, la velocidad de Fermi \(v_{F}\) es

                        \begin{align}
                            v_{F} &= \dfrac{\hbar}{m}(3\pi^{2}n)^{\slashfrac{1}{3}},\label{eq:FermiVelocityComplete}\\
                            &= \dfrac{\qty{1.054e-34}{\J\s}}{\qty{9.109e-31}{\kg}}\cdot (3\pi^{2}(\qty[per-mode=symbol]{8.45e28}{\atoms\per\m\cubed}))^{\slashfrac{1}{3}},\nonumber\\
                            \Aboxedmain{v_{F} &= \qty[per-mode=symbol]{1.57e6}{\m\per\s}.}\label{eq:FermiVelocityForCopper}
                        \end{align}

                        Si comparamos la velocidad de arrastre con la velocidad de Fermi notaremos que esta última es 9 ordenes de magnitud mayor.
                    \end{solution}
                    
                    \color{blue}
                    \item ¿Qué magnitud tendría que tener dicho campo eléctrico para que \(v_{a}\) y \(v_{F}\) fueran iguales? Escribe tu resultado en unidades de \unit{\V\per\m}.
                    \color{black}
                    \begin{solution}
                        Para determinar la magnitud de \(E\) igualamos \cref{eq:DriftSpeed} con \cref{eq:FermiVelocity} y resolvemos para éste,

                        \begin{align*}
                            \dfrac{\sigma E}{n\e} &= v_{F},\\
                            E &= v_{F}\dfrac{n\e}{\sigma}.
                        \end{align*}
                        
                        Tal que

                        \begin{align*}
                            E &= (\qty[per-mode=symbol]{1.57e6}{\m\per\s})\dfrac{(\qty[per-mode=symbol]{8.45e28}{\atoms\per\m\cubed})(\qty{1.6e-19}{\C})}{\qty[per-mode=power]{5.9e7}{\per\ohm\per\m}},\\
                            \Aboxedmain{E &\simeq \qty[per-mode=symbol]{3.59e8}{\V\per\m}.}
                        \end{align*}
                    \end{solution}
                    
                    \color{blue}
                    \item Estima el valor del camino libre medio \(\ell\) para el cobre a \qty{300}{\kelvin} y compara este valor con el espaciamiento promedio de los átomos del metal.
                    \color{black}
                    \begin{solution}
                        De \cref{eq:Conductivity} sabemos que el valor del camino libre medio \(\ell\) está dado por

                        \begin{equation*}
                            \ell = v_{F}\dfrac{\sigma m}{n\e^{2}}.
                        \end{equation*}

                        Tal que,

                        \begin{align*}
                            \ell &= (\qty[per-mode=symbol]{1.57e6}{\m\per\s})\dfrac{(\qty[per-mode=power]{5.9e7}{\per\ohm\per\m})(\qty{9.109e-31}{\kg})}{(\qty[per-mode=symbol]{8.45e28}{\atoms\per\m\cubed})(\qty{1.6e-19}{\C})^{2}},\\
                            \Aboxedmain{\ell &\simeq \qty{3.9e-8}{\m} = \qty{390}{\angstrom}.}
                        \end{align*}

                        Recordamos que el valor de espaciamiento promedio entre los átomos a temperatura ambiente es de \qty{400}{\angstrom}, por lo que el valor de \(\ell\) que se obtuvo para el cobre a temperatura ambiente es prácticamente el mismo.
                    \end{solution}
                \end{enumerate}

                \color{blue}
                Información útil para resolver este problema: el cobre es un metal monovalente, lo que significa que hay un electrón libre por cada átomo ( o sea, cada átomo del metal dona un electrón de conducción). La densidad de átomo del cobre es \(n = \qty{8.45e28}{\atoms\per\m\cubed}\). La conductividad eléctrica del cobre a \qty{300}{\K} es \(\sigma = \qty[per-mode=power]{5.9e7}{\per\ohm\per\m}\)
        \end{enumerate}
    \end{exercise}
\end{document}